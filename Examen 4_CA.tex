\documentclass[12pt]{exam}
\usepackage[utf8]{inputenc}		% Caracteres latinos
\usepackage[spanish]{babel}		% Idioma español
\usepackage{geometry}			% Organizar el documento
\usepackage{graphicx}			% Incluir gráficos
\usepackage{makecell}			% Para personalizar las celdas de una tabla
\usepackage[nohdr]{mathexam}	% Añadimos el paquete mathexam (sin header)
\usepackage{amsmath}
\usepackage{amsfonts}
\usepackage{amssymb}
\usepackage{mathtools}
\usepackage{tikz,pgfplots}
\usepgfplotslibrary{polar}
\usepackage[shortlabels]{enumitem}
 \renewcommand{\baselinestretch}{1.5}
\usepackage{mathtools}
\usepackage{bm}
\usepackage{esvect}
\usepackage[fleqn]{mathtools}
\usepackage{relsize}
\usepackage{multirow}
\usepackage{multicol}
\usepackage[document]{ragged2e}
 \usepackage{textpos}
\usepackage{tcolorbox}
\usepackage{hyperref}


\geometry{
	a4paper,                    % Tamaño del documento
	hmargin = {1.7cm, 1.7cm}, 	% Margen horizontal izquierdo, derecho
	vmargin = {1cm, 1cm},	    % Margen vertical superior, inferior
	headsep = 4mm,				% Separación entre el encabezado y el texto
	head = .2cm,				% Tamaño del encabezado
	% marginparsep = 5mm, 		% Seperación entre las notas y el texto
	% marginpar = 1.5cm,		% Tamaño de las notas
	includeall,                 % incluye el encabezado, footer y notas dentro del tamaño del documento
	nomarginpar,	            % Elimina las notas
	foot = 1cm,                 % Tamaño del footer
	twoside,                	% Habilita el modo de impresión a doble cara
}

\selectlanguage{spanish}       
\spanishdecimal{.}


\newcommand{\iuni}{\pmb{\hat{\imath}}}
\newcommand{\juni}{\pmb{\hat{\jmath}}}
\newcommand{\kuni}{\pmb{\hat{k}}}
% DOCUMENTO
\begin{document}

\centering


\Large 
\textbf{Examen 4}\\
%\large 
%Unidad 2: Integrales triples. Integrales de línea\\
%Valor: 25 puntos\\


\normalsize

\pointpoints{punto}{puntos}
\pointformat{\bfseries\boldmath(\thepoints)}
\vskip12pt

    
    \begin{questions}

     \question (2.5 pts.)
       Si la proporción de nacimientos de una población es $b(t)=2200e^{0.024t}$ personas por cada año y la de decesos es $d(t)=1460e^{0.018t}$ personas por cada año. Encuentra el área entre estas curvas para $0\leq t\leq 10$. ¿Qué representa el área?
    
\vskip10pt
\question (3 pts.)
El administrador de un restaurante de comida rápida determina que el tiempo promedio que sus clientes esperan a ser atendidos es 2.5 minutos.
\begin{enumerate}[a)]
\item Encuentra la probabilidad (usando una función de densidad de probabilidad exponencial) de que un cliente tenga que esperar durante más de 4 minutos.
\item Encuentra la probabilidad de que un cliente sea atendido
dentro de los primeros dos minutos.
\item El administrador quiere anunciar que cualquier persona que no sea atendida dentro de cierto número de minutos, tiene derecho a una hamburguesa gratis. Pero no quiere dar hamburguesas gratis a más de 2\% de sus clientes. ¿Qué debe decir el anuncio?
\end{enumerate}
\vskip10pt
%    \question  (2.5 pts.) Si $f(t)$ es continua para $t\geq 0$, la \textit{transformada de Laplace de $f$} es la función de $F$ definida por $$F(s)=\displaystyle\int_0^\infty f(t)e^{-st}\;dt$$
%    y el dominio de $F$ es el conjunto de todos los números $s$ para los que la integral converge. Encuentra las transformadas de Laplace de las siguientes funciones:
    
 %   \begin{enumerate}[a)]
%        \item $f(t)=e^t$.
 %       \item $f(t)=t$.
%    \end{enumerate}{}

    \question  (2.5 pts.) Si $f(t)$ es continua para $t\geq 0$, la \textit{transformada de Laplace de $f$} es la función de $F$ definida por $$F(s)=\displaystyle\int_0^\infty f(t)e^{-st}\;dt$$
    y el dominio de $F$ es el conjunto de todos los números $s$ para los que la integral converge. Encuentra la transformada de Laplace de la  función $f(t)=t$

\vskip10pt

 \question  (2 pts.) La respiración es cíclica y un ciclo respiratorio completo —desde el principio de la inhalación hasta el final de la exhalación— requiere alrededor de 5 $s$. La tasa máxima de aire que entra en los pulmones es de más o menos 0.5 $L/s$, por lo que a menudo se ha usado la función $f(t)=\dfrac{1}{2}sen\left(\dfrac{2\pi t}{5}\right)$ para modelar la razón de flujo de aire hacia los pulmones. Usa este modelo para determinar el volumen de aire inhalado en los pulmones al tiempo $t$.
\vskip10pt
\textbf{Punto extra}: La densidad lineal de una varilla de 8 $m$ de longitud es $\rho(x)=\dfrac{12}{\sqrt{x+1}} \, kg/m$, donde $x$ se mide en metros desde un extremo de la varilla. Determina la densidad promedio de la varilla. 

%Determina si la siguiente integral converge o diverge. $$\displaystyle\int_1^\infty \dfrac{1}{\sqrt{x^2 -0.1}}\;dx$$

%\question Si $f(x)=(1+x)^{-1}$, encuentra una fórmula para la $n-$ésima derivada $f^{(n)}(x)$.

    
        \end{questions}
        \vskip30pt
 \RaggedRight
     
    \newpage


\newgeometry{
	hmargin = {1.5cm, 1.5cm},
	vmargin = {5cm, 1cm},
	nohead,			% Elimina el encabezado
	nomarginpar,	% Elimina las notas
	includeall,
}% \savegeometry{geometria_1}

\pagestyle{foot}    % El estilo de ésta página sólo constará de pié de página
\runningfooter{}{}{Página \thepage\ de \numpages}

\end{document}
