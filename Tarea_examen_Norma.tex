\documentclass {article} 
\usepackage [spanish] {babel} 
\usepackage [T1]{fontenc}
\usepackage [latin1]{inputenc}
\usepackage{graphicx}
\usepackage{amsmath}
\usepackage{amsmath, amssymb, latexsym}
\usepackage{amsthm}
\usepackage{amsfonts}
\usepackage[dvips]{epsfig}
\usepackage{indentfirst}
%\usepackage[retainorgcmds]{IEEEtrantools}
\usepackage[all]{xy}
\usepackage{tikz}
%\usepackage{makeidx}
\addtolength{\hoffset}{-2cm}
\addtolength{\textwidth}{4cm}
\addtolength{\voffset}{-2.5cm}
\addtolength{\textheight}{5cm}
\usetikzlibrary{arrows,shapes,matrix,decorations,shapes.geometric}
\usepackage{shortvrb}
\pagestyle{empty}
\usepackage{mathrsfs}
%opening
\title{}
\author{}

\begin{document}

\begin{center}
{\LARGE \bf  Tarea-Examen de Derivadas}\\

\

Profesora: Norma Yanet S\'anchez Torres.

Ayudante: Alejandra Carolina Barrios Barocio.\\

\

%\begin{IEEEeqnarray*}{rCl}
%AYUDANTES &:& \textnormal{Patricio Garc\'ia,}\\
%& &\textnormal{Judith V\'azquez,}\\
%& &\textnormal{Alejandra Carolina Barrios.}\\
%\end{IEEEeqnarray*}
\end{center}


\begin{enumerate}

\item Con ayuda de diagramas, dibujos, fotograf\'ias, video, etc., trata de encontrar una forma en c\'omo le explicar\'ias a uno de tus compa\~neros de matem\'aticas II el concepto de derivada por que tuvo problemas y no logr\'o llegar a la clase en donde se explic\'o dicho concepto (usa tu ingenio), no quiero copias de internet (utiliza tus propias palabras), si realizas un video puedes enviarlo por correo electr\'onico.


\item \textquestiondown En qu\'e puntos la gr\'afica de la funci\'on $g(x)=x^3-3x$ tiene tangentes horizontales?

\item Encuentra las ecuaciones de todas las l\'ineas que tienen pendiente $-1$ y que son tangentes a la curva $y=1/(x-1)$.

\item Demuestra que si $f(x)=cos(x)$ entonces $f'(x)=-sen(x)$ usando la definici\'on formal de derivada.

\item \textquestiondown Cu\'al es la raz\'on de cambio del \'area de un c\'irculo ($A=\pi r^2$) con respecto al radio cuando el radio es $r=3$?

\item Cuando se a\~nadi\'o un bactericida a un caldo nutritivo en el que crec\'ian las bacterias, la poblaci\'on de bacterias continu\'o creciendo durante un tiempo, pero luego dej\'o de crecer y comenz\'o a disminuir. El tama\~no de la poblaci\'on al tiempo $t$ (horas) fue $b=10^6+10^4t-10^3t^2$. Encuentra las tasas de crecimiento  en 

\begin{itemize}
\item[a.] $t=0$ horas.
\item[b.] $t=5$ horas.
\item[c.] $t=10$ horas.
\end{itemize}

\item Si un gas en un recipiente cerrado se mantiene a temperatura $T$ constante , la presi\'on $P$ se relaciona con el volumen $V$ mediante la f\'ormula de la forma: $(P+\frac{an^2}{V^2})(V-nb)=nRT$, en la cual $a$, $n$, $b$ y $R$ son constantes, encuentra $\frac{dP}{dV}$.

\item Encuentra las derivadas de las siguientes funciones paso a paso y menciona las reglas de derivaci\'on que usaste en cada paso.
  
\begin{itemize}
\item [a.] $y=csc x-4\sqrt{x}+7$
\item [b.] $y=(sec x+tan x)(sec x-tan x)$
\item [c.] $y=x^2 sen x+2x cos x-2 senx$
\item [d.] $y=\frac{1+csc x}{1-csc x}$
\item [e.] $y=\frac{sen x+cos x}{cos x}$
\end{itemize}


\item La posici\'on de una part\'icula que se mueve a lo largo de una recta es $s=\sqrt{1+4t}$, con $s$ en metros y $t$ en segundos. Encuentra la velocidad de la part\'icula en $t=6$ segundos.  

\item Obtener las derivadas de las siguientes funciones:

\begin{itemize}
\item [a.] $f(x)=(\frac{x}{5}+\frac{1}{5x})^5$
\item [b.] $f(x)=sec(tan x)$
\item [c.] $f(x)=sen (\frac{3\pi x}{2})+cos (\frac{3\pi x}{2})$
\item [d.] $f(x)=(\frac{sen x}{1+cos x})^2$
\item [e.] $f(x)=4 sen(\sqrt{1+\sqrt{x}})$
\end{itemize}


\item Dibuje la gr\'afica de la funci\'on $f(x)=x^4-8x^2+16$ en el intervalo $[-5,4)$, encuentre los puntos cr\'iticos, los extremos relativos usando la prueba de la primera derivada y determine los extremos absolutos de la funci\'on en el intervalo. Finalmente, determine los intervalos en los que la funci\'on es creciente o decreciente.

\item La reacci\'on del cuerpo a una dosis de medicina puede algunas veces ser representada por una ecuaci\'on de la forma $$R=M^2(\frac{C}{2}-\frac{M}{3})$$ donde $C$ es una constante positiva y $M$ es la cantidad de medicina absorbida en la sangre. Si la reacci\'on es un cambio en la presi\'on de la sangre, $R$ se mide en mil\'imetros de mercurio. Si la reacci\'on es un cambio de temperatura, $R$ se mide en grados, etc. \\


Encuentra $dR/dM$. Esta derivada, como una funci\'on de $M$, se conoce como la sensibilidad del cuerpo a la medicina. Usa esta \'ultima para encontrar la cantidad de medicina para la cual el cuerpo es m\'as sensible. Para esto, hay que encontrar el valor de $M$ que maximize la derivada $dR/dM$.

\end{enumerate}

\

\

NOTA: Recuerda que la tarea se puede hacer en equipo, pero la entrega es individual. Tambi\'en recuerda que es una tarea-examen, por lo tanto debes explicar con detalle todos tus pasos, no omitas ning\'un comentario ya que son necesarios para entender qu\'e es lo que estas haciendo, tampoco omitas ninguno de los pasos algebr\'aicos.


\end{document}