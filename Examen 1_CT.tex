\documentclass[12pt]{exam}
\usepackage[utf8]{inputenc}		% Caracteres latinos
\usepackage[spanish]{babel}		% Idioma español
\usepackage{geometry}			% Organizar el documento
\usepackage{graphicx}			% Incluir gráficos
\usepackage{makecell}			% Para personalizar las celdas de una tabla
\usepackage[nohdr]{mathexam}	% Añadimos el paquete mathexam (sin header)
\usepackage{amsmath}
\usepackage{amsfonts}
\usepackage{amssymb}
\usepackage{mathtools}
\usepackage{tikz,pgfplots}
\usepgfplotslibrary{polar}
\usepackage[shortlabels]{enumitem}
 \renewcommand{\baselinestretch}{1.5}
\usepackage{mathtools}
\usepackage{bm}
\usepackage{esvect}
\usepackage[fleqn]{mathtools}
\usepackage{relsize}
\usepackage{multirow}
\usepackage{multicol}
\usepackage[document]{ragged2e}
 \usepackage{textpos}
\usepackage{tcolorbox}
\usepackage{hyperref}


\geometry{
	a4paper,                    % Tamaño del documento
	hmargin = {1.7cm, 1.7cm}, 	% Margen horizontal izquierdo, derecho
	vmargin = {1cm, 1cm},	    % Margen vertical superior, inferior
	headsep = 4mm,				% Separación entre el encabezado y el texto
	head = .2cm,				% Tamaño del encabezado
	% marginparsep = 5mm, 		% Seperación entre las notas y el texto
	% marginpar = 1.5cm,		% Tamaño de las notas
	includeall,                 % incluye el encabezado, footer y notas dentro del tamaño del documento
	nomarginpar,	            % Elimina las notas
	foot = 1cm,                 % Tamaño del footer
	twoside,                	% Habilita el modo de impresión a doble cara
}

\selectlanguage{spanish}       
\spanishdecimal{.}


\newcommand{\iuni}{\pmb{\hat{\imath}}}
\newcommand{\juni}{\pmb{\hat{\jmath}}}
\newcommand{\kuni}{\pmb{\hat{k}}}
% DOCUMENTO
\begin{document}

\centering


\Large 
\textbf{Examen 2}\\
%\large 
%Unidad 2: Integrales triples. Integrales de línea\\
%Valor: 25 puntos\\


\normalsize

\pointpoints{punto}{puntos}
\pointformat{\bfseries\boldmath(\thepoints)}
\vskip15pt

    
    \begin{questions}

     \question (3 pts.) Resuelve únicamente un problema:
 \begin{enumerate}[I)]
   \item La reacci\'on del cuerpo a una dosis de medicina puede algunas veces ser representada por una ecuaci\'on de la forma $$R=M^2(\frac{C}{2}-\frac{M}{3})$$ donde $C$ es una constante positiva y $M$ es la cantidad de medicina absorbida en la sangre. Si la reacci\'on es un cambio en la presi\'on de la sangre, $R$ se mide en mil\'imetros de mercurio. Si la reacci\'on es un cambio de temperatura, $R$ se mide en grados, etc. \\
    Encuentra $dR/dM$. Esta derivada, como una funci\'on de $M$, se denomina la sensibilidad del cuerpo a la medicina. Usa esta \'ultima para encontrar la cantidad de medicina para la cual el cuerpo es m\'as sensible. 

    \item Una plataforma de perforación a 12 km de la costa debe ser conectada mediante un oleoducto a una refinería que está a 20 km en línea recta desde el punto de la costa más cercano a la plataforma. Si instalar la tubería debajo del agua cuesta \$500,000 por km y en tierra cuesta \$300,000 por km, ¿Qué combinación de instalación subacuática y terrestre da la conexión más barata?
\end{enumerate}
    
\vskip10pt
%    \question (7 pts.)
%    Si $f(x)=ax^3+bx^2+cx+d$, determina $a$, $b$, $c$, $d$ de modo que $f$ tenga un extremo relativo en el punto $(0,3)$ y un punto de inflexión en $(1,-1)$. Dibuja la gráfica resultante.
     
\vskip10pt
%     \question (6 pts.)
%       Sea $f(x)$ continua en $[a,b]$ y diferenciable en $(a,b)$. Demuestra que, si $f'(x)<0$ para toda $x\in (a.b)$ entonces $f(x)$ es decreciente en $[a,b]$.

%\vskip10pt
%     \question (6 pts.)
%     Se va a fabricar una lata para que contenga 1 L de aceite. Encuentra las dimensiones que minimizarán el costo del metal para fabricar la lata. 
  
    
%\vskip10pt
\question (2.5 pts.)
    Encuentra una función cúbica $f(x)=ax^3+bx^2+cx+d$ que tenga un máximo local en el punto $(3,-2)$ y un mínimo local en el punto $(1,0)$. 

%\question (2.5 pts.) Considere una barra de 80 cm. Suponga que la temperatura es de 500 °C en el punto 0, y 495 °C en $x = 20$. Suponga que la temperatura está dada por la fórmula $T(x) = 500-ax$ para toda $x$ de 0 a 80. ¿Cuánto debe valer $a$? Determine la diferencia en temperatura entre el punto $x = 30$ y $x = 40$. ¿Cuál es el promedio en el decremento de la temperatura en el intervalo $[30,40]$? Calcule la tasa de cambio instantánea de la temperatura en $x = 30$ y para cualquier $x$.

    \question  (2.5 pts.) Se forma una molécula del producto $C$ a partir de una molécula del reactivo $A$ y una molécula del reactivo $B$ y las concentraciones iniciales de $A$ y $B$ tienen un valor común $[A]=[B]=a\,moles/L$, entonces
    $$[C]=\frac{a^2kt}{akt+1}$$
    donde $k$ es una constante
    \begin{enumerate}[i)]
        \item Determina la velocidad de reacción en el instante $t$.
        \item Demuestra que si $x=[C]$, entonces $\frac{dx}{dt}=k(a-x)^2$.
    \end{enumerate}{}

\question (2 pts.) Si $f(x)=x^n$, encuentra una fórmula para la $n-$ésima derivada $f^{(n)}(x)$.

%\question Si $f(x)=(1+x)^{-1}$, encuentra una fórmula para la $n-$ésima derivada $f^{(n)}(x)$.

\vskip15pt
\textbf{Punto extra}: Derivar $f(x)=(x^3 -1)^{100}$

    
        \end{questions}
        \vskip30pt
 \RaggedRight
     
    \newpage


\newgeometry{
	hmargin = {1.5cm, 1.5cm},
	vmargin = {5cm, 1cm},
	nohead,			% Elimina el encabezado
	nomarginpar,	% Elimina las notas
	includeall,
}% \savegeometry{geometria_1}

\pagestyle{foot}    % El estilo de ésta página sólo constará de pié de página
\runningfooter{}{}{Página \thepage\ de \numpages}

\end{document}